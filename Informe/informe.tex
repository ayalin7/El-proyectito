\documentclass{article}
%\usepackage[spanish]{babel} 
%\selectlanguage{english}  
\usepackage[utf8]{inputenc} 
\usepackage[T1]{fontenc} 
\usepackage{array}
\usepackage{geometry}
 \geometry{
 a4paper,
 total={170mm,257mm},
 left=20mm,
 top=20mm,
 }
\usepackage{amsmath, amsthm, amsfonts} 
\usepackage{graphicx} 
\usepackage[colorinlistoftodos]{todonotes} 
\usepackage[colorlinks=true, allcolors=blue]{hyperref} 
\usepackage[square,numbers]{natbib}
\usepackage{upgreek}
\usepackage{lipsum}  

\title{Informe de Laboratorio Grupo Bacan } 
\author{Benjam\'in Ayala Baeza  \\ \href{mailto:beayala2021@udec.cl}{beayala2021@udec.cl} 
        \and Mar\'ia Ignacia Espinoza Inzunza \\ \href{mailto:marespinoza2021@udec.cl}{marespinoza2021@udec.cl} \\
        \and Florencia Fuentes Jara \\ \href{mailto:flofuentes2021@udec.cl}{flofuentes2021@udec.cl} \\
        \and Mart\'in Sepulveda Zu\~niga \\ \href{mailto:msepulveda2021@udec.cl}{msepulveda2021@udec.cl} \\
\date{Universidad de Concepci\'on \\ \today}
}
\setlength{\marginparwidth}{2cm}
\begin{document}
\maketitle

\begin{abstract}
En el presente documento expondremos la relación entre la fuerza dependiente de la distancia y el giro de un cuerpo. Trataremos este sistema con su variable dependiente 
\end{abstract}

\section{Introducción} \label{intro}
 A lo largo de este informe expondremos nuestros procedimiento, resultados y análisis para nuestro proyecto de laboratorio del curso Laboratorio I haciendo uso de los conocimientos adquiridos durante el curso.\\
 Nuestra premisa para este experimento consta de estudiar de una manera eficaz los efectos de la aplicación de una fuerza tangencial a un objeto que posee una capacidad de rotación. Basaremos nuestro estudio sobre el fenómeno del torque y del momento angular (juntos con su conservación) y como este ultimo principio no es tan efectivo ante las condiciones definidas en la literatura. Tal como veremos en nuestro experimento, notaremos que las condiciones representadas en las ecuaciones y los libros son bastante lejanas a las condiciones que veremos en la practica, esto nos obliga a trabajar con margenes de error, ajustes para los datos de forma que podamos estudiar la tendencia teórica con la practica. 
\section{Objetivos}
Nuestros objetivos a lograr con esta actividad de laboratorio son :

\begin{itemize}
    \item Observar número de vueltas realizadas respecto a la longitud de la cinta métrica.
    \item Salvar al mundo del calentamiento global 

\end{itemize}

 \section{Experimentación}
Para encontrar la correlación entre distancia y cantidad de vueltas, debemos repetir el experimento en diversas condiciones. Nuestra condición variable corresponde a la distancia desde la cual veremos como afecta la fuerza con la cual la cinta métrica se cierra.
    \subsection{Materiales}
    En este proyecto utilizamos los siguientes instrumentos:
    \begin{itemize}
        \item Cinta Métrica 3m
        \item Papel con ejes de referencia
        \item Cinta adhesiva
        \item Celular con cámara lenta
    \end{itemize}
    \subsection{Procedimiento}
    A continuación describiremos detalladamente el paso a paso realizado para las diversas mediciones hechas:
        \begin{enumerate}
            \item Comenzamos colocando la cinta métrica al centro del papel con ejes de referencia, de forma que si extendemos la cinta métrica esta crece a lo largo del eje de las ordenadas. La cinta métrica y debe estar pegada al papel, esto lo logramos usando la cinta adhesiva.
            \item Alineamos en nuestra mesa el papel con ejes de referencia y proyectamos estos ejes en la mesa, de forma que tengamos una extensión de estos para observar cuanto gira el montaje de la cinta métrica con el papel. 
            \item Posicionamos nuestro celular con cámara lenta por encima del montaje del papel, mesa y cinta de forma que tengamos una visión área del sistema.
            \item Tomamos el comienzo de la cinta métrica y extendemos hasta llegar a los $10 [cm]$, soltamos la cinta y vemos cuanto gira. Todo este proceso mientras grabamos. 
            \item Repetimos el paso 4. veinte veces y anotamos los resultados.
            \item Repetimos todo el procedimiento para la medida de la cinta métrica igual a 20 [cm], 30 [cm], 40 [cm], 50 [cm], 60 [cm], 70 [cm], 80 [cm], 90 [cm] y 100 [cm].
            
        \end{enumerate}



\section{Resultados}

\section{Análisis}


\section{Conclusión}


\bibliographystyle{apalike}
\bibliography{Informe/referencias.bib}
\end{document}